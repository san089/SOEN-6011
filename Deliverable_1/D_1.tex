\documentclass[a4paper,12pt]{article}
\usepackage[left=0.75in,right=0.75in,top=1.5in,bottom=1.5in,footskip=.25in]{geometry}
\usepackage{graphicx}
\usepackage[english]{babel}
\usepackage[utf8x]{inputenc}
\usepackage{longtable}
\usepackage{url}

\newcommand{\Author}{Sanchit Kumar(40081187)}
\newcommand{\ProjectName}{Function 9: Power Function}
\newcommand{\Title}{}
\newcommand{\Course}{\textbf{SOEN-6011 \\ Software Engineering Processes}}
\newcommand{\ProfessorName}{Dr. PANKAJ KAMTHAN}

\graphicspath{{D:/SOEN 6011/SOEN-6011/Deliverable_1}}

\begin{document}

%----------------------------------------------------------------------------------------
%	TITLE PAGE START
%----------------------------------------------------------------------------------------

\begin{titlepage}
\newcommand{\HRule}{\rule{\linewidth}{0.5mm}} %New command for thickness of line	


\centering
\textsc{\LARGE Concordia University} \\ [5mm] 
\includegraphics[scale=.1]{University_logo.jpg}\\[1cm] 
\textsc{\Large \Course} \\ [0.5cm]

%--------
%	TITLE
%--------
	
\HRule \\[0.4cm]
{ \huge \bfseries SCIENTIFIC CALCULATOR \\ [5mm]  \ProjectName}\\[0.4cm] 
{\large \textbf{DELIVERABLE 1 (D1)} }	
\HRule \\[1.5cm]


%---------
%	TAIL SECTION
%---------
\vspace{5cm}
\Large \emph{\textbf{Author: \Author}}\\
{\large \today}\\[2cm]

\vfill
\end{titlepage}	





\newpage
%----------------------------------------------------------------------------------------
%	NEW PAGE START
%----------------------------------------------------------------------------------------

\section{\large Function 9 : Power Function}

\subsection{Description}
A power function is of the form:
\begin{equation} \label{Power_func}
f(x) = x^y
\end{equation}
where y is a real number.


\subsection{Domain}
\begin{enumerate}
	\item When y is a non-negative integer, the domain is all real numbers:  (- $\infty$,$\infty$)
	
	\item When y is a negative integer, the domain is all real numbers excluding zero ( (- $\infty$, 0) $\cup$ (0,$\infty$) )
	
	\item When y is a irrational number and y \textgreater \space 0, the domain is all non-negative real numbers.
	
	\item When y is a irrational number and y \textless \space 0, the domain is all positive real numbers. 
\end{enumerate}


\subsection{Characteristics of Power Function.}
\begin{enumerate}
	\item The behaviour of power function depends on whether the y is a positive or a negative number.
	\item The behaviour of power function depends on whether the y is even or odd.
	\item Also, the power function behaves differently for fractional powers and specifically for negative or positive fractional powers.
\end{enumerate}


%----------------------------------------------------------------------------------------
%	Requirement Specifications
%----------------------------------------------------------------------------------------


\section{Requirements Specification}
\subsection{Definitions and abbrevations}


\begin{table}[htp]
	\centering
	\caption{Definitions and abbreviations.} \vspace{0.5cm} \label{tab:definition_table} 
	\begin{tabular}{||c|c||}
		\hline  \hline \textbf{Terms} & \textbf{Definition} \\
		\hline \hline
		FR & Funtional Requirement  \\ 
		\hline
		NFR & Non-Functional Requirement  \\
		\hline
		User & Someone who interacts with the system. \\
		\hline
		System & Software Program for calculation of Power Function. \\
		\hline \hline 
	\end{tabular}
\end{table}



\subsection{Constraints and Assumptions}
\begin{enumerate}
	\item User should provide input for both "x" and "y". No default values to be used.
	\item Based on function characteristics, value of "x" and "y" should be a real number.
	\item If input value for "x" is less than 0, then "y" is a whole number.
	\item The output is contrained by Hardware. 
	\item The maximum value program could calculate is 3.40282346638528860e+38.
	\item The minimum value program could calculate is -3.40282346638528860e+38.
\end{enumerate}








\section{Requirememts}

\subsection{Functional Requirements}
\begin{itemize}
	
	\item \textbf{ID } \hspace{3cm} :FR1  \\
	\textbf{TYPE } \hspace{2.27cm}  :Functional\\
	\textbf{PRIORITY } \hspace{1.15cm} :1 \\
	\textbf{DESCRIPTION }\hspace{0.35cm} :System should prompt the user to enter the value of x and y. \\
	\textbf{RATIONALE } \hspace{0.75cm} :In order to get user input and start calculation. \\
	
	
	\item \textbf{ID } \hspace{3cm} :FR2  \\
	\textbf{TYPE } \hspace{2.27cm}  :Functional\\
	\textbf{PRIORITY } \hspace{1.15cm} :1 \\
	\textbf{DESCRIPTION }\hspace{0.35cm} :System should display an error message when value entered by user is not a number. \\
	\textbf{RATIONALE } \hspace{0.75cm} :For calculations, input should be numbers only.  \\
	
	
	\item \textbf{ID } \hspace{3cm} :FR3  \\
	\textbf{TYPE } \hspace{2.27cm}  :Functional\\
	\textbf{PRIORITY } \hspace{1.15cm} :2 \\
	\textbf{DESCRIPTION }\hspace{0.35cm} :In case the input entered is not valid system should prompt the user to input values again. \\
	\textbf{RATIONALE } \hspace{0.75cm} :User should have the flexiblity to do calculations without exiting the program.  \\
	
	
	\item \textbf{ID } \hspace{3cm} :FR4  \\
	\textbf{TYPE } \hspace{2.27cm}  :Functional\\
	\textbf{PRIORITY } \hspace{1.15cm} :2 \\
	\textbf{DESCRIPTION }\hspace{0.35cm} :User should have the option to exit the program anytime during the use. \\
	\textbf{RATIONALE } \hspace{0.75cm} :If user is done with the use of program.  \\
	
	
	\item \textbf{ID } \hspace{3cm} :FR5  \\
	\textbf{TYPE } \hspace{2.27cm}  :Functional\\
	\textbf{PRIORITY } \hspace{1.15cm} :1 \\
	\textbf{DESCRIPTION }\hspace{0.35cm} :System should display an error message when user enter value of x and y both as 0. \\
	\textbf{RATIONALE } \hspace{0.75cm} :0 raised to the power 0 is undefined.  \\
	
	
	\item \textbf{ID } \hspace{3cm} :FR6  \\
	\textbf{TYPE } \hspace{2.27cm}  :Functional\\
	\textbf{PRIORITY } \hspace{1.15cm} :1 \\
	\textbf{DESCRIPTION }\hspace{0.35cm} :System should display an error message when user enter value of x as 0 and y as a negative number. \\
	\textbf{RATIONALE } \hspace{0.75cm} :0 raised to the power of a negative number is undefined.  \\
	

\end{itemize}



\subsection{Non-Functional Requirements}
\begin{itemize}
	
	\item \textbf{ID } \hspace{3cm} :NFR1  \\
	\textbf{TYPE } \hspace{2.27cm}  :Non-Functional\\
	\textbf{PRIORITY } \hspace{1.15cm} :3 \\
	\textbf{DESCRIPTION }\hspace{0.35cm} :The error message displayed should be appropriate and helpful for the user. \\
	\textbf{RATIONALE } \hspace{0.75cm} :User should be able to know what went wrong.  \\
	
	
	\item \textbf{ID } \hspace{3cm} :NFR2  \\
	\textbf{TYPE } \hspace{2.27cm}  :Non-Functional\\
	\textbf{PRIORITY } \hspace{1.15cm} :3 \\
	\textbf{DESCRIPTION }\hspace{0.35cm} :The text-based interface should be user friendly. \\
	\textbf{RATIONALE } \hspace{0.75cm} :It should be easy for the user to use the system.  \\
	
	
	\item \textbf{ID } \hspace{3cm} :NFR3  \\
	\textbf{TYPE } \hspace{2.27cm}  :Non-Functional\\
	\textbf{PRIORITY } \hspace{1.15cm} :2 \\
	\textbf{DESCRIPTION }\hspace{0.35cm} :The result displayed should be as accurate as possible. \\
	\textbf{RATIONALE } \hspace{0.75cm} :Incorrect output should not be displayed.   \\
	
	
	\item \textbf{ID } \hspace{3cm} :NFR4  \\
	\textbf{TYPE } \hspace{2.27cm}  :Non-Functional\\
	\textbf{PRIORITY } \hspace{1.15cm} :3 \\
	\textbf{DESCRIPTION }\hspace{0.35cm} :Calculation time should be less than 1 second. \\
	\textbf{RATIONALE } \hspace{0.75cm} :Waiting a long time for the output might not be desirable for the user.  \\
	
	
\end{itemize}


\subsection{Difficulty and Prioritization}

%----------------------------------------------------------------------
%               Difficulty and Priority Table
%----------------------------------------------------------------------

\begin{table}[htp]
	\centering
	\caption{Difficulty and Prioritization} \vspace{0.5cm} \label{tab:difficulty_table} 
	\begin{tabular}{||c|c|c||}
		\hline  \hline \textbf{Requirement ID} & \textbf{Priority} & \textbf{Difficulty} \\
		\hline \hline
		FR1 & High & Easy \\ 
		\hline
		FR2 & High & Easy \\ 
		\hline
		FR3 & Normal & Easy  \\ 
		\hline
		FR4 & Normal & Normal \\ 
		\hline
		FR5 & High & Easy  \\ 
		\hline
		FR6 & High & Normal \\ 
		\hline
		NFR1 & Low & Easy  \\ 
		\hline
		NFR2 & Low & Normal  \\ 
		\hline
		NFR3 & Normal & Difficult  \\ 
		\hline
		NFR4 & Low & Normal  \\ 
		\hline
	\end{tabular}
\end{table}



%----------------------------------------------------------------------------------------
%	Psudo Code
%----------------------------------------------------------------------------------------













\begin{thebibliography}{3}
	
	\bibitem{1} \emph{“Power Functions Algebraic Representation”} \url{http://wmueller.com/precalculus/families/1\textunderscore41.html} 
	
	\bibitem{2} \emph{"Power Functions"} \url{http://www.biology.arizona.edu/biomath/tutorials/Power/Powerbasics.html}
	
\end{thebibliography}

%-------------------------------PAGE END-------------------------------------------------------------------------------------------------------------

\end{document}