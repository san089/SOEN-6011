%Problem Statement 1

\documentclass[a4paper,12pt]{article}
\usepackage[margin=1.00in]{geometry}

\title{Problem 1\\
		\large Course - SOEN 6011, Professor - Pankaj Kamthan}
\author{Sanchit Kumar, ID - 40081187}



\begin{document}

\maketitle %To display the title in the document.

\section{\large Function 9 : Power Function}

\subsection{Description}
A power function is of the form:
\begin{equation} \label{Power_func}
	f(x) = x^y
\end{equation}
where y is a real number.


\subsection{Domain}
\begin{enumerate}
\item When y is a non-negative integer, the domain is all real numbers:  (- $\infty$,$\infty$)

\item When y is a negative integer, the domain is all real numbers excluding zero ( (- $\infty$, 0) $\cup$ (0,$\infty$) )

\item When y is a irrational number and y \textgreater \space 0, the domain is all non-negative real numbers.

\item When y is a irrational number and y \textless \space 0, the domain is all positive real numbers. 
\end{enumerate}


\subsection{Characteristics of Power Function.}
\begin{enumerate}
\item The behaviour of power function depends on whether the y is a positive or a negative number.
\item The behaviour of power function depends on whether the y is even or odd.
\item Also, the power function behaves differently for fractional powers and specifically for negative or positive fractional powers.
\end{enumerate}



\end{document}